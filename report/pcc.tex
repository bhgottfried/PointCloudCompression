\documentclass[twoside,twocolumn]{article}

\usepackage[sc]{mathpazo} % Use the Palatino font
\usepackage[T1]{fontenc} % Use 8-bit encoding that has 256 glyphs
\linespread{1.05} % Line spacing - Palatino needs more space between lines
\usepackage{microtype} % Slightly tweak font spacing for aesthetics

\usepackage[english]{babel} % Language hyphenation and typographical rules

\usepackage[hmarginratio=1:1,margin=20mm,top=20mm,columnsep=20pt]{geometry} % Document margins
\usepackage[hang, small,labelfont=bf,up,textfont=it,up]{caption} % Custom captions under/above floats in tables or figures
\usepackage{booktabs} % Horizontal rules in tables

\usepackage{enumitem} % Customized lists
\setlist[itemize]{noitemsep} % Make itemize lists more compact

\usepackage{abstract} % Allows abstract customization
\renewcommand{\abstractnamefont}{\normalfont\bfseries} % Set the "Abstract" text to bold
\renewcommand{\abstracttextfont}{\normalfont\itshape} % Set the abstract itself to italic text

\usepackage{amsmath, amsfonts, amsthm}  % Math packages
\newcommand{\symdiff}{\ensuremath{\bigtriangleup}}
\newcommand{\Symdiff}{\raisebox{-1.2ex}{\Huge \ensuremath{\bigtriangleup}}}

\usepackage{algorithmicx}     % Algorithm pseudocode packages
\usepackage[ruled]{algorithm}
\usepackage[noend]{algpseudocode}
\algnewcommand{\LineComment}[1]{\State \(\triangleright\) #1} % Command for line comments
\algnewcommand\algorithmicforeach{\textbf{foreach}}  % command for foreach loop
\algdef{S}[FOR]{ForEach}[1]{\algorithmicforeach\ #1\ \algorithmicdo}

\usepackage{listings} % Display data structures
\lstset{language=C} % Set the struct language

\usepackage{titling} % Customizing the title section
\newcommand\textline[4][t]{ % Brute force command to center author names
  \par\smallskip\noindent\parbox[#1]{.333\textwidth}{\raggedright#2}%
  \parbox[#1]{.333\textwidth}{\centering#3}%
  \parbox[#1]{.333\textwidth}{\raggedleft#4}\par\smallskip%
}

\usepackage{hyperref} % For hyperlinks in the PDF

%----------------------------------------------------------------------------------------
%	TITLE SECTION
%----------------------------------------------------------------------------------------

\setlength{\droptitle}{-4\baselineskip} % Move the title up

\pretitle{\begin{center}\Huge\bfseries} % Article title formatting
\posttitle{\end{center}} % Article title closing formatting
\title{Point Cloud Compression Optimization} % Article title

% % I'm sorry for this...
% \author{
% \textline[t]{$\ $\textsc{Milind Kulkarni}}
% {\textsc{Benjamin Gottfried}\thanks{Ben Gottfried performed this work as an undergraduate at Purdue
% University. He is seeking admission to CS PhD programs, as well as internships for Summer 2021.}$\quad\ $}
% {\textsc{Kirshanthan Sundararajah}}\\[0.5ex]
% \textline[t]{\normalsize Purdue University, USA}
% {\normalsize Purdue University, USA$\quad\ $}
% {\normalsize Purdue University, USA$\qquad\ \ \  $}\\[-1ex]
% \textline[t]{$\ \ \,$\normalsize \href{mailto:milind@purdue.edu}{milind@purdue.edu}}
% {\normalsize \href{mailto:bg@purdue.edu}{bg@purdue.edu}$\quad\ $}
% {\normalsize \href{mailto:ksundar@purdue.edu}{ksundar@purdue.edu}$\qquad\quad\ \,$}
% }

\author{
\textsc{Benjamin Gottfried}\thanks{Ben Gottfried performed this work as an undergraduate at Purdue
University. He is seeking admission to CS PhD programs, as well as internships for Summer 2021.}\\[1ex]
\normalsize Purdue University, USA\\
\normalsize \href{mailto:bg@purdue.edu}{bg@purdue.edu}
\and
\textsc{Kirshanthan Sundararajah}\\[1ex]
\normalsize Purdue University, USA\\
\normalsize \href{mailto:ksundar@purdue.edu}{ksundar@purdue.edu}
\and
\textsc{Milind Kulkarni}\\[1ex]
\normalsize Purdue University, USA\\
\normalsize \href{mailto:milind@purdue.edu}{milind@purdue.edu}
}

\date{} % Leave empty to omit a date
\renewcommand{\maketitlehookd}{%
\begin{abstract}
\noindent This paper improves upon previous methods to encode and compress point cloud streams by exploiting
spatial and temporal redundancy in point data. When implemented, a point cloud stream can be represented
by a series of octrees that represents the 3-dimensional objects in the scene captured in each frame.
The effectiveness of the previously presented encoding method is examined, and its computational efficiency
is analyzed and improved. This paper presents methods of exploiting task and data parallelism that found
in various procedures used in the serial implementation of the encoding algorithm.
\end{abstract}
}

%----------------------------------------------------------------------------------------

\begin{document}

\maketitle % Print the title

%----------------------------------------------------------------------------------------
%	ARTICLE CONTENTS
%----------------------------------------------------------------------------------------

\section{Introduction} \label{sec:Introduction}

Point clouds are almost ubiquitously used in any application involving representing spatial data in three
dimensions. Hence, they are paramount to many applications in graphics, for example, representing a football
field, on which to perform calculations and draw additional objects, such as the line of scrimmage.
One of the most prolific and intuitive uses of point clouds is in the sensor data generated by the
Microsoft Kinect.

The Kinect continuously scans a room and represents the scene as a 3-dimensional space at a rate of
30 frames per second. Each frame captured can be represented as a set of points, where each point $(x,y,z)$
represents whether or not some object is occupying the space centered at that point at that time frame.
The set of points at each time frame make up a point cloud, and when multiple, consecutive point clouds
are chained together, the resulting data structure is referred to as a point cloud stream.

%------------------------------------------------

\subsection{Data Representation} \label{sec:Data Representation}

There already exists a convenient file format (.pcd) for defining point clouds. This file consists
of a small header, which we will mostly ignore, along with a set of $n \, (x,y,z)$ coordinates
that make up the cloud. We can represent this point cloud as an octree, where each octree node corresponds
with a bounded, 3-dimensional space (a voxel). In this paper, "voxel," "octree," and "octree node," will be
used almost interchangeably.

The root of the octree is the entirety of the 3-dimensional space represented by the scene, and each
child maps to a partitioned subvoxel of its parent voxel. The 8 children of this octree correspond to 8
subvoxels that equally partition the space (not contained points) contained by the parent. These
subvoxels are recursively partitioned until the leaf nodes are at the pre-defined maximum recursion-depth.
Hence, all leaf nodes (resp. voxels) are at the same depth (resp. size). This is necessary for the
correctness of our algorithms. Each subvoxel is bounded by its parent voxel and the midpoints between
each maximum and minimum $x, y,$ and $z$ coordinate that bounds the parent voxel.

Each octree node exists if and only if there exists at least one point from the original point set
that is contained within its voxel's bounds, otherwise that specific node is null. Each octree node contains
an array of pointers to its children, which are null if their subvoxels are unpopulated. Hence, the data
contained at each octree node can be efficiently represented as an 8-bit integer, where each bit is set if
and only if the corresponding child pointer exists. The struct representing an octree node is as follows:

\begin{lstlisting}
  typedef struct _OctreeNode {
    struct _OctreeNode* children[8];
    char data;
  } OctreeNode;
\end{lstlisting}

Let $d$ be the maximum recursion-depth. Given a point set, an octree can be constructed in
$O(nd)$ time, and $O(8^d)$ space. This construction is performed by recursively inserting each point $p$
into the subvoxel that bounds $p$, setting the data bit in the current node that corresponds to the
newly populated subvoxel.

\alglanguage{pseudocode}
\begin{algorithm}[h]
\small
\caption{Create an octree by recursively inserting each point, creating nodes and setting bits as necessary}
\label{alg:CreateOctree}
\begin{algorithmic}[1]
\Procedure {$\mathbf{CreateOctree}$} {PointSet $P$}
  \Procedure {$\mathbf{InsertPoint}$} {OctreeNode $T$, Point $p$}
    \If {$T$ is not a leaf node}
      \LineComment{Find the suboctant $c$ that contains $p$}
      \LineComment{Set the bit in $T.\mathtt{data}$ corresponding to $c$}
      \State $\mathbf{InsertPoint}(c, p)$
    \EndIf
  \EndProcedure

  \State OctreeNode $T$
  \ForEach{(Point $p$ : $P$)}
    \State $\mathbf{InsertPoint}(T, p)$
  \EndFor
  \State \Return T
\EndProcedure
\Statex
\end{algorithmic}
  \vspace{-0.4cm}
\end{algorithm}

If the only important data associated with a point cloud is whether or not a given point in space is
populated, there exists an efficient way to compress a set of $(x,y,z)$ coordinates. The compression
will lose some small details, like how many points exist in each leaf voxel, or other metadata
associated with a point, like color, but in the case of our Kinect example, these losses are acceptable,
provided our recursion depth is deep enough to allow a sufficiently granular resolution to make decisions
based on the positions of objects.

In \textbf{Section \ref{sec:Stream Compression}}, we will argue why it is most efficient to maintain the
relative bounds of each captured scene dynamically, rather than keeping maximum and minimum bounds fixed.
Thus, the bounds of the scene associated with each point cloud are defined by the maximum and minimum
$x, y,$ and $z$ coordinates, plus and minus approximately 5\%, respectively, so that no point lies
exactly on the borders defining the bounds.

%----------------------------------------------------------------------------------------

\section{Octree Serialization} \label{sec:Octree Serialization}

The octree representing the point cloud can be serially encoded using the following algorithm:
Informally, first append the scene's bounds $\vec{B}$ to a byte list $S$, then perform
a breadth-first traversal of all non-leaf nodes, appending the data byte of each node visited to $S$.

\alglanguage{pseudocode}
\begin{algorithm}[h]
\small
\caption{Serialize an octree into a byte list}
\label{alg:Serialize}
\begin{algorithmic}[1]
\Procedure {$\mathbf{Serialize}$} {OctreeNode $T$}
  \State ByteList $S$
  \State $S$.append($\vec{B}$)
  \State Queue $Q$
  \State $Q$.enqueue($T$)
  \While {($\neg Q$.empty())}
    \State OctreeNode $t \gets Q$.dequeue()
    \If {($t$ is not a leaf node)}
      \State $S$.append($t.\mathtt{data}$)
      \ForEach {(OctreeNode $c : t.\mathtt{children}$)}
        \State $Q$.enqueue($c$)
      \EndFor
    \EndIf
  \EndWhile
  \State \Return $S$
\EndProcedure
\Statex
\end{algorithmic}
  \vspace{-0.4cm}
\end{algorithm}

The serialized octree can then be decoded by essentially performing the inverse of
\textbf{Algorithm \ref{alg:Serialize}}. Each byte $b$ creates a new tree node $t$ with $b$ as its data
byte, which gets traversed breadth-first.

Any octree can then be restored into a sufficiently accurate point cloud (.pcd file) by assigning a point
to either the centroid or a specific corner of every populated leaf voxel.

\alglanguage{pseudocode}
\begin{algorithm}[h]
\small
\caption{Decode an octree from a byte list}
\label{alg:Deserialize}
\begin{algorithmic}[1]
\Procedure {$\mathbf{Deserialize}$} {ByteStream $S$}
  \State Octree $T$
  \State Queue $Q$
  \State $Q$.enqueue($T$)
  \ForEach {(Byte $b$ : $S$)}
    \State OctreeNode $t \gets Q$.dequeue()
    \State $t.data \gets b$
    \ForEach{(set bit $i$ : $b$)}
      \State OctreeNode $t_i$
      \State $Q$.enqueue($t_i$)
      \State $t.\mathtt{children}[i] \gets t_i$
    \EndFor
  \EndFor
  \State \Return T
\EndProcedure
\Statex
\end{algorithmic}
\vspace{-0.4cm}
\end{algorithm}

If defining a point cloud as an octree with a maximum resolution has sufficient precision, and if
the only pertinent data is the presence of an object at a location, then serializing the octree
as described is almost always a more efficient use of space. We will argue this claim in the general case,
but first, consider the format of point cloud data in applications like the Kinect, or a football
game. These intuitions will further justify why a compression scheme like this is optimal.

Most of the space in any scene is vacuous. There are only 1-4 players standing in a room, or moreover,
2 small teams taking up an entire football field. This leads to fairly sparse data, but for the objects
that do exist, they are usually contiguous, uniform chunks of mass. This implies that the overwhelming
majority of the voxels for the data we are compressing fall into one of two cases:

\begin{itemize}
  \item The voxel is completely empty, so a large volume of space, or an octree node with shallow depth,
  is empty, so it is written off and ignored.
  \item There exists an object in a mid-level octree node, whose suboctants are nearly all populated.
  A contiguous object implies that the leaf voxel for the center of that object is populated, as well as
  a sizable number of adjacent leaf voxels are all populated. The data bits for parent voxels
  containing this object are comprised of almost entirely '1' bits, and the least common ancestor,
  or the smallest containing subvoxel, of the populated leaf nodes would be relatively small.
\end{itemize}

Using this intuition, we will now prove the claim that even the uncompressed serialization is more
space efficient than the standard point cloud format (.pcd), which uses a set of floating point
$x, y,$ and $z$ coordinates. Let $C_P$ denote a point cloud represented as a .pcd file, and let $C_S$
denote the same file, but serialized using \textbf{Algorithm \ref{alg:Serialize}}. The size of a single point in $C_P$
is always a constant 12 bytes, assuming the size of a float is 32 bits, but the total size of the set
of points in $C_S$ varies. We assume that our scene is comprised of two types of objects: larger objects
of contiguous mass, like humans or furniture, and isolated objects that only populate a small number
of adjacent leaf voxels, like balls in a game.

For a single, isolated point in $C_S$, there are $d$ bytes that recursively partition the space until the
leaf voxel is defined as populated. In the worst case, if our scene consisted of exclusively small, isolated
points, $|C_P| = 12n$ and $|C_S| = dn$, so our encoded octree would be larger than the .pcd file by a factor
bounded by $\frac{d}{12}$. If $d < 12$, it is already more efficient to store $C_S$, regardless of how the
data is laid out.

% This should probably be a lot more succinct and elegant...

But for a larger, contiguous object, the overhead of the bytes representing internal nodes is
amortized because the least common ancestor of adjacent points is just their parent, so the bytes
in their serialization are only written once. Let $x$ be a larger, contiguous object. $C_P$ must
contain at least as many points to comprise $x$ as there are populated leaf voxels in $C_S$. Suppose
to the contrary that $C_P$ uses fewer points to represent $x$ than $C_S$. Then there is at least one
subvoxel in $C_S$ that contains a point not in $C_P$. However, we defined $C_S$ to be the serialization
of $C_P$, and using \textbf{Algorithm \ref{alg:Serialize}}, a leaf voxel is never created unless there
exists a point bounded exactly by that leaf voxel; a contradiction. Hence, the number of points in $C_P$
that compose $x$ $|x \subset C_P| \geq |x \subset C_S|$, and by our assumption that $x$ is a larger
object of contiguous points, the overhead to get the least common ancestor of the points in
$x \subset C_S$ is shared by those points, so the factor $\frac{d}{12}$ is amortized.

As $d$ increases, the overhead for an isolated point increases, but by our assumption that $x$ exists
in $C_P$, which implies $x$ exists in $C_S$, as $d$ increases, $C_P$ must contain even more points
to achieve the same level of granularity. Since $x$ is contiguous and it must have as many as points
$C_S$, the depth of the least common ancestor of $x \subset C_S$ is constant, so the compression ratio
only increases.

%----------------------------------------------------------------------------------------

\section{Stream Compression} \label{sec:Stream Compression}

In this section, we will present a serial algorithm for encoding a point cloud stream \cite{pcsc},
rather than sending each point cloud as either a .pcd file, or even a serialized octree as presented
in \textbf{Section \ref{sec:Octree Serialization}}.

We introduce the notion of taking the symmetric difference between two octrees. Intuitively, the
symmetric difference between two trees is the set of nodes that exist in only one of the two trees.
Voxels that are either vacant or populated in both are not included in the symmetric difference.
Why this operation is useful will become clear shortly. First, we present an algorithm that takes
two octrees $T_i, T_j$ and calculates the symmetric difference between the two. The result returned
is a list of bytes, where a '0' bit represents no change from the previous tree, and a '1' bit
represents a tree node that in $T_j$ that was not in $T_i$, or vice versa. Notationally,
$$T_i \symdiff T_j = \delta_{i,j}$$

% I'm not sure of the best way to do the foreach loop in this algorithm...

\alglanguage{pseudocode}
\begin{algorithm}[h]
\small
\caption{Calculate the symmetric difference between 2 octrees}
\label{alg:CalcSymDiff}
\begin{algorithmic}[1]
\Procedure {$\mathbf{CalcSymDiff}$} {OctreeNode $T_i$, OctreeNode $T_j$}
  \State ByteList $\delta$
  \State Queue $Q$
  \State $Q$.enqueue($(T_i, T_j)$)
  \While {($\neg Q$.empty())}
    \State $t_i, t_j \gets Q$.dequeue()
    \If {($t_i, t_j$ are not leaf nodes)}
      \State $\delta$.append($t_i.\mathtt{data} \oplus t_j.\mathtt{data}$)
      \State $\vec{C} \gets t_i.\mathtt{children} \ \mathtt{OR} \ t_j.\mathtt{children}$
      \ForEach {(OctreeNode $(c_i, c_j) : \vec{C}$)}
        \State $Q$.enqueue($(c_i, c_j)$)
      \EndFor
    \EndIf
  \EndWhile
  \State \Return $\delta$
\EndProcedure
\Statex
\end{algorithmic}
\vspace{-0.4cm}
\end{algorithm}

This algorithm is correct in its computation of $\delta_{i,j}$ since every populated voxel in both trees
$T_i, T_j$ is visited during the breadth-first traversal exactly once. Upon visitation, the data bytes of
the current nodes in both trees are $\mathtt{XOR}$ed, yielding the change between those nodes, which is the
essence of symmetric difference between octrees. Furthermore, if one tree node does not exist in either
$T_i$ or $T_j$, then the data byte of the tree node that does exist is $\mathtt{XOR}$ed with 0, yielding
itself. This is intuitively correct since if a subtree did not previously exist, or a subtree newly exists,
that subtree will need to change completely from the previous tree. Hence, by our definition of
symmetric difference, \textbf{Algorithm \ref{alg:CalcSymDiff}} is correct.

The key intuition behind the algorithm for encoding a series of
point clouds is that the objects in the scene are changing very little between two consecutive frames.
Hence, the bits that are set, respectively cleared, in $T_i$ are mostly the same as in $T_{i+1}$.
From this observation, we can deduce that the byte list $\delta_{i,j}$ will be almost entirely '0' bits.
This lends itself extremely well to run-length encoding to further compress the size.

Let us also revisit the idea touched upon at the end of \textbf{Section \ref{sec:Data Representation}}
that it is beneficial to dynamically update the bounds of the scene for every point cloud. A pattern that
might frequently occur in these 3-dimensional data-capturing sensors is translations. A sensor might pan
around a football game, or a person playing Kinect might jump. The relative locations of all objects in
the scene are changing with respect to the sensor, but their orientations are mostly static. Consequently,
by storing the maximum and populated coordinate points in each dimension, a box can be maintained around
the objects in focus, and relative to that focus box, the objects are not changing. This means that despite
the scene changing, we store a constant value with our encoded scene (2 floats per dimension), but our
symmetric difference focus boxes between adjacent frames is still mostly '0' bits, which maintains the
effectiveness of RLE compression.

The following is the complete algorithm for encoding and compressing a stream of point clouds. Informally,
we store the encoded initial scene $T_0$ as a byte list, then for each subsequent point set $P_i$, create
an octree $T_i$, write its bounds $\vec{B}_i$, calculate the symmetric difference between the previous and
current trees $\delta_{p,i} = T_p \symdiff T_i$, compress $\delta_{p,i}$ using RLE, then write it.

\alglanguage{pseudocode}
\begin{algorithm}[h]
\small
\caption{Compress an incoming point cloud stream}
\label{alg:Stream Compression}
\begin{algorithmic}[1]
\Procedure {$\mathbf{StreamCompress}$} {PointSet$[\,]$ $P$}
  \State OctreeNode $T_0 \gets \mathbf{CreateOctree}(P_0)$
  \State ByteList $S_0 \gets \mathbf{Serialize}(T_0)$
  \State \textbf{Write}($S_0$)
  \State OctreeNode $T_p \gets T_0$
  \ForEach {(PointSet $P_i : P$)} \Comment{as new $P_i$ stream in}
    \State OctreeNode $T_i \gets \mathbf{CreateOctree}(P_i)$
    \State \textbf{Write}($\vec{B}_i$)
    \State $\delta_{p,i} \gets \mathbf{CalcSymDiff}(T_p, T_i)$
    \State \textbf{Write}($\mathbf{RLE}(\delta_{p,i})$)
    \State $T_p \gets T_i$
  \EndFor
\EndProcedure
\Statex
\end{algorithmic}
\vspace{-0.4cm}
\end{algorithm}

%------------------------------------------------

\subsection{Decompression} \label{sec:Decompression}

We have shown that this algorithm is a space-efficient way to encode a point cloud stream, but the encoding
is useless if we cannot recover the original point sets. Accordingly, we need to show that the original
point sets can be reconstructed using only our encoded stream. Let the number of additional point sets
contained in the stream after $P_0$ be $n$. Since, any octree can be restored into a sufficiently
accurate point cloud, to show that each points et $P_i, i \leq n$ can be reconstructed, it suffices to show
that each corresponding octree $T_i, i < n$ can be recovered from the encoding.
We prove this by strong induction:

For the base case, we will show that $T_1$ can be reconstructed from $T_0$. First, reacquire $T_0$ by
performing \textbf{Algorithm \ref{alg:Deserialize}}. Next, decode $\delta_{0,1}$ by undoing the RLE.
Now, assume there exists an algorithm to reconstruct an octree $T_i$ given the previous octree
$T_p, p = i - 1$ and the symmetric difference between the two octrees $\delta_{p,i} = T_p \symdiff T_i$.
If this algorithm does exist, then we have everything we need to use it to reconstruct $T_1$. Therefore,
our base case holds.

For the inductive step, we assume that we can reconstruct all trees $T_r, 1 \leq r \leq k$. To discharge
our inductive hypothesis, we must show that it is possible to reconstruct $T_{k+1}$. As long as
$k + 1 \leq n$, $\delta_{k,k+1}$ is somewhere in our encoded byte stream. Since we assumed $T_k$ can be
reconstructed, we can use the aforementioned algorithm to reconstruct $T_{k+1}$ using $T_k$ and
$\delta_{k,k+1}$. Hence, our inductive case holds and any tree $T_i$ can be reconstructed
$\forall i, i \leq n$.

In order to complete this proof, we need to show that the previously utilized, but undefined,
adjacent reconstruction algorithm exists and is correct. Let $T_p, T_i$ be the previous octree and the
octree currently being constructed, respectively. Informally, iterate across each byte $b \in \delta_{p,i}$,
while performing a breadth-first traversal of both $T_p$ and $T_i$. Let the currently visited node
of $T_p$ (resp. $T_i$) be $t_p$ (resp. $t_i$). Then $t_i.\mathtt{data}$ is $b \oplus t_p.\mathtt{data}$.
Next, iterate across each bit in $b$ and the data byte of $t_p$.
If only one of the bits is set at a position $j$ in both bytes, then $t_i$ has a child at position $j$.
Otherwise, the child of $t_i$ at position $j$ is null. Irrespective of the nullity of both the previous
and current tree's children at $j$, enqueue the children, or null in its place, then continue the
traversal for all $b \in \delta_{p,i}$.

\alglanguage{pseudocode}
\begin{algorithm}[h]
\small
\caption{Reconstruct an Octree given the previous octree and their symmetric difference}
\label{alg:Reconstruct}
\begin{algorithmic}[1]
\Procedure {$\mathbf{Reconstruct}$} {OctreeNode $T_p$, ByteList $\delta_{p,i}$}
  \State OctreeNode $T_i$
  \State Queue $Q$
  \State $Q$.enqueue($(T_p, T_i)$)
  \ForEach {(byte $b$ : $\delta_{p,i}$)}
    \State $t_p, t_i \gets Q$.dequeue()
    \State $t_i.\mathtt{data} \gets b \oplus t_p.\mathtt{data}$
    \State byte $c \gets b \ \mathtt{OR} \ t_p.\mathtt{data}$
    \ForEach {(set bit $j : c$)}
      \State OctreeNode $t_j \gets b[j] \oplus t_p.\mathtt{data}[j]$
      \State $t_i.\mathtt{children}[j] \gets t_j$
      \State $Q.$enqueue($(t_p.\mathtt{children}[j], t_j)$)
    \EndFor
  \EndFor
  \State \Return $T_i$
\EndProcedure
\Statex
\end{algorithmic}
\vspace{-0.4cm}
\end{algorithm}

Before we prove \textbf{Algorithm \ref{alg:Reconstruct}} to be correct in the general case, let us consider
the special case of where the previous tree is null. This is a non-issue, since a null octree is
treated the same as an octree with no populated voxels. This would lead to every set bit in the
symmetric difference byte list to correspond with a populated node in the new tree. Hence, it can be seen
that $\delta_{\emptyset,k} = S_k = \mathbf{Serialize}(T_k)$. We now prove this algorithm in the general
case to be correct by induction:

% Help formatting this? I don't like the way the math looks at the end of the paragraph
For the base case, consider two octrees $T_i, T_j$ of depth $d = 1$, such that, by our definition of depth,
all of $T_x$'s children, $x \in \{i,j\}$ are leaf voxels. This implies that their full serializations
using \textbf{Algorithm \ref{alg:Serialize}} are just $T_x.\mathtt{data}$, $x \in \{i,j\}$.
Hence, $\delta_{i,j} = T_i \symdiff T_j = T_i.\mathtt{data}$ $\oplus T_j.\mathtt{data}$,
where $|\delta_{i,j}|$ = 1 byte. Let that byte be $b$. By definition of
\textbf{Algorithm \ref{alg:CalcSymDiff}}, the only set bits in $b$ are the positions $j$, such that
$T_x.\mathtt{children}[j]$ exists for one and only one $x \in \{i,j\}$.
Accordingly, running \textbf{Algorithm \ref{alg:Reconstruct}} with $T_i$ and $\delta_{i,j}$ as input
yields a single octree node $T_k$, such that $T_k.\mathtt{data} = T_i.\mathtt{data} \oplus b$.\\
but $b = T_i.\mathtt{data} \oplus T_j.\mathtt{data}$, so\\
$T_k.\mathtt{data} = T_i.\mathtt{data} \oplus T_i.\mathtt{data} \oplus T_j.\mathtt{data}
= T_j.\mathtt{data}$\\
which implies $T_k = T_j$.
Therefore, the base case holds.

For the inductive case, assume a tree $t_j$ of depth $d = k$ can be reconstructed, given a tree $t_i$,
also of depth $d = k$, and $\delta_{i,j} = t_i \symdiff t_j$. We will use this assumption to then show
that given a tree $T_i$ of depth $k + 1$ and its respective $\delta_{i,j} = T_i \symdiff T_j$, we can
reconstruct $T_j$. By our inductive hypothesis, a tree of depth $d = k$ can be reconstructed. The subtree
of $T_j$ consisting of only the internal nodes $T_x$ is a an octree of depth $d = k$, so we can we
reconstruct that. During the execution of \textbf{Algorithm \ref{alg:Reconstruct}}, for all nodes
considered, their children are enqueued, so there exists a bijective mapping of all of the remaining
bytes in $\delta_{i,j}$ not used to reconstruct $T_x$ to the tree nodes in $T_i \setminus T_x$.
All of the nodes in $T_i \setminus T_x$ are all of depth $d = k + 1$, so they are all parents of leaf
nodes of $T_i$. These individual parents of leaf nodes nodes are also subtrees of depth $d = 1$, so using
their mapped byte, their individual subtrees can be reconstructed in the same manner as the base case.
Since $T_x$ can be reconstructed, and all of the subtrees coming from the leaves of $T_x$ can be
reconstructed, then $T_j$ can be reconstructed, discharging our inductive hypothesis.
Therefore $\forall d, d \geq 0$, trees of depth $d$ can be reconstructed, so
\textbf{Algorithm \ref{alg:Reconstruct}} is correct.

This completes the proof for the feasibility and correctness for our stream decompression algorithm.
$\square$

%------------------------------------------------

\section{Parallel Optimizations} \label{sec:Parallel Optimizations}

The majority of the paper thus far has been discussion and formal proof of previously known and tested
serial algorithms. We now look toward throughput-oriented hardware to extract more performance from the
spatial and temporal redundancies in point cloud streams, but we must first redesign our algorithms
accordingly.

%------------------------------------------------

\subsection{Parallel Compression} \label{sec:Parallel Compression}

Since our compression algorithm is a streaming algorithm, it will always be bottlenecked by the rate at
which new data points enter the compression system. If one had all the entire series of point sets at the
beginning, then obviously the entire encoding and compression process could happen simultaneously for all
point sets, since all of the data sets are independent. But we will not assume that impracticality, thus
maintaining the streaming capabilities of our compression algorithm. Consequently, our high-level
compression procedure is identical to that presented in \textbf{Algorithm \ref{alg:Stream Compression}},
but we can take advantage of our additional threads to further optimize the subroutines utilized.

The first potential optimization we would like to explore is constructing the octree in parallel. I argue
this is possible by first stably sorting all points in $x,y,$ then $z$ order, partitioning them into each
subvoxel of the root in constant time, setting the data bit for subvoxel with at least one point, then
recursing. Each thread would be responsible for performing the same task on exactly one subvoxel, so
this exhibits great data parallelism.

\alglanguage{pseudocode}
\begin{algorithm}[h]
\small
\caption{Create an octree in parallel by sorting and partitioning all points, setting data bits, then recursing}
\label{alg:ParCreateOctree}
\begin{algorithmic}[1]
\Procedure {$\mathbf{ParCreateOctree}$} {PointSet $P$}
  \State OctreeNode $T$
  \LineComment {Stably sort all $p \in P$ by $(x,y,z)$ coordinates}
  \State $\vec{P}_c \gets \textbf{Partition}(P)$
  \State $T.\mathtt{data} \gets |\vec{P}_c|$
  \ForEach {(PointSet $P_c : \vec{P}_c$)}
    \State $T$.children$[c] \gets \mathbf{spawn \ ParCreateOctree}(P_c)$
  \EndFor
  \State $\mathbf{sync}$
  \State \Return T
  \EndProcedure
\Statex
\end{algorithmic}
  \vspace{-0.4cm}
\end{algorithm}

This parallel elision of \textbf{Algorithm \ref{alg:CreateOctree}} is still correct because each after
being partitioned, each thread operates on a disjoint set of points, since each point can exist on only
one partition. This algorithm would undoubtedly be less efficient if executed sequentially, but since
each subvoxel can be considered simultaneously, this is a performance upgrade.

The next improvement made is one to the subroutine that encodes the serialization of an octree,
\textbf{Algorithm \ref{alg:Serialize}}. The optimization takes advantage of parallel breadth-first
traversal paradigm first designed by Jo et al \cite{trees}. The essence of the paradigm is each
computation performed at every level of the tree happens simultaneously, syncing all threads before
recursing to the next level. Each thread locally executes in a depth-first manner, which further
improves upon the spacial locality often missing in serial breadth-first traversals.

Since the same operations are all performed in each computation, this optimization exploits great data
parallelism. The soundness of the following algorithmic paradigm is implicit because it has been shown
that the parallel execution of breadth-first search holds using this technique. Since
\textbf{Algorithm \ref{alg:Serialize}} is just a form of breadth first traversal over an octree,
the paradigm can be applied here. Since the following algorithm is non-trivial and relies on a few newly
presented concepts, the following paragraphs will outline \textbf{Algorithm \ref{alg:ParSerialize}} before
pseudocode is given.

First, we must define the new notion of a position sequence. The position sequence is the string of indices
required to get from the root of the octree to the node for which the position sequence corresponds. Thus,
the position sequence of the root is the empty string $\lambda$, and the number of characters in the
position sequence is its depth. Let us also define the serialization $Z_l$ for an individual level $l$
of the octree. Since \textbf{Algorithm \ref{alg:Serialize}} executes breath-first, it is impossible to
have a serialization that contains a data byte from a level $x$ prior to a byte from a level $y$, such
that $x$ is deeper in the octree than $y$. $Z_l$ can also be defined as the data bytes of all tree nodes on
level $l$, sorted with respect to each node's position sequence. Hence, the full serialization $Z$ of an
octree of depth $d$ is simply $\sum_{l = 0}^d Z_l$, where concatenation is used instead of addition.
This notion of $Z_l$ will become important later.

\textbf{Algorithm \ref{alg:ParSerialize}} accepts a thread block as input, where each thread contains an
octree, a position sequence (which will be soon be defined), and a set containing bytes and position
sequences.  Before handling the individual threads, the algorithm must create a new thread block
$B_n$ to contain the threads for all of the nodes in the next deepest level of the tree. It also creates
a new set, called the FlatSet $F$, which contains the sets of bytes for all of the threads in the current
thread block.

\textbf{Algorithm \ref{alg:ParSerialize}} then executes all of the threads in the thread block
simultaneously for all of the nodes in the thread block's level of the tree, performing the following
additional computations: Let $t$ be an arbitrary thread, and let $T, S, P$ be $t$'s octree, byte set, and
position sequence, respectively. First, $T$'s data and $P$ are added to $S$. Then, $S$ is added to $F$.
Note that all set add operations must be synchronized, so a mutex lock or another synchronization
primitive must be used. Next, $t$ creates its own empty byte set $S_t$. For each populated child
$c$ of $T$, a new thread is added (but not yet executed) to $B_n$ that consists of $c, S_t$, and the
concatenation of $P$ with the index used to get $c$.

After performing computations on each thread in the previous thread block,
\textbf{Algorithm \ref{alg:ParSerialize}} computes in parallel the following: It recurses, calling itself
again with $B_n$ as its input. Additionally, it computes the result of a "flatten" operation on $F$.
Let this operation be $\mathbf{Flatten}(F)$ and let it return $Z_l$, where $l$ is the level of the octree
on which $F$ is defined. We will prove this to be feasible in the following paragraph. $\mathbf{Flatten}$
returns $Z_l$, so the result of $\mathbf{Flatten}$ can be concatenated with the output of the next recursive
call to the algorithm to yield $\sum_{l = 0}^d Z_l = Z$ which is the full serialization of $T$.

In order to prove the correctness of \textbf{Algorithm \ref{alg:ParSerialize}}, we must define the
internals of $\mathbf{Flatten}$. $F$ is a set of sets, such that the union of all its members contain all
of the data bytes for the octree level $l$ for which $F$ is defined. Sorting those data bytes in order
of their position sequences yields $Z_l$. Therefore,
$$\bigcup_{S \in F} S = \{b | b \in Z_l\} \rightarrow \mathbf{sorted}(\bigcup_{S \in F} S) = Z_l$$
which implies, concatenating a series of calls to $\mathbf{Flatten}$, one for each level of the octree,
yields the full serialization of the octree:
$$\sum_{l = 0}^d Z_l = Z$$

\alglanguage{pseudocode}
\begin{algorithm}[h]
\small
\caption{Serialize an octree in parallel}
\label{alg:ParSerialize}
\begin{algorithmic}[1]
  \LineComment {Each thread contains: OctreeNode, PositionSeq, ByteSet}
  \Procedure {$\mathbf{ParSerialize}$} {ThreadBlock $B$}
    \State ThreadBlock $B_n$
    \State FlatSet $F$
    \ForEach {(Thread $t : B$)}
      \If {($t.T$ is not a leaf node)}
        \State $t.S.$add($t.T.\mathtt{data}, t.P$)
        \State $F$.add($t.S$)
        \State ByteSet $S_t$
        \ForEach {(OctreeNode $c : t.T.$children)}
          \State $B_n.$add(Thread($c, S_t, t.P + c.\mathtt{pos}$)
        \EndFor
      \EndIf
    \EndFor
    \State \Return $\mathbf{Flatten}(F) + \mathbf{ParSerialize}(B_n)$
  \EndProcedure
\Statex
\end{algorithmic}
  \vspace{-0.4cm}
\end{algorithm}

% Yea, I have a lot of different algorithm numbers now... How can I make this more clear?
An analogy can be drawn between \textbf{Algorithm \ref{alg:Serialize}} and
\textbf{Algorithm \ref{alg:CalcSymDiff}}, since the internals of the algorithms for serialization and
symmetric differentiation are very similar --- the key difference being in
\textbf{Algorithm \ref{alg:Serialize}}, only the raw data byte is appended, whereas in
\textbf{Algorithm \ref{alg:CalcSymDiff}}, the two tree nodes' data bytes are $\mathtt{XOR}$ed. The same
analogy can be drawn between \textbf{Algorithm \ref{alg:ParSerialize}}. Since
\textbf{Algorithm \ref{alg:CalcSymDiff}} is also a breadth-first traversal, we can apply this parallelized
breadth-first traversal paradigm here, as well, to further optimize our speed of computing the symmetric
difference between two octrees. \textbf{Algorithm \ref{alg:ParSymDiff}} is correct for the same reasons 
\textbf{Algorithm \ref{alg:ParSerialize}} is correct, along with the same reasons why
\textbf{Algorithm \ref{alg:CalcSymDiff}} works as a serial, breadth-first traversal.

\alglanguage{pseudocode}
\begin{algorithm}[h]
\small
\caption{Calculate the symmetric difference between 2 octrees in parallel}
\label{alg:ParSymDiff}
\begin{algorithmic}[1]
\LineComment {Each thread contains: OctreeNode ($i,j$), PositionSeq, ByteSet}
\Procedure {$\mathbf{ParSymDiff}$} {ThreadBlock $B$}
  \State ThreadBlock $B_n$
  \State FlatSet $F$
  \ForEach {(Thread $t : B$)}
    \If {($t.T_i$ and $t.T_j$ are not a leaf nodes)}
      \State $t.S.$add($t.T_i.\mathtt{data} \oplus t.T_j.\mathtt{data}, t.P$)
      \State $F$.add($t.S$)
      \State ByteSet $S_t$
      \State $\vec{C} \gets t.T_i.\mathtt{children} \ \mathtt{OR} \ t.T_j.\mathtt{children}$
      \ForEach {(OctreeNode $(c_i, c_j) : \vec{C}$)}
        \State $B_n.$add(Thread($(c_i, c_j), S_t, t.P + c.\mathtt{pos}$)
      \EndFor
    \EndIf
  \EndFor
  \State \Return $\mathbf{Flatten}(F) + \mathbf{ParSymDiff}(B_n)$
\EndProcedure
\Statex
\end{algorithmic}
\vspace{-0.4cm}
\end{algorithm}

Using our parallelized subroutines, the improved complete algorithm for encoding and compressing a stream
of point clouds, which is \textbf{Algorithm \ref{alg:Stream Compression}} with its serial subroutines
replaced, is \textbf{Algorithm \ref{alg:ParStreamCompress}}.

\alglanguage{pseudocode}
\begin{algorithm}[h]
\small
\caption{Compress an incoming point cloud stream in parallel}
\label{alg:ParStreamCompress}
\begin{algorithmic}[1]
\Procedure {$\mathbf{StreamCompress}$} {PointSet$[\,]$ $P$}
  \State OctreeNode $T_0 \gets \mathbf{ParCreateOctree}(P_0)$
  \State ThreadBlock $B \gets $ThreadBlock$(T_0, \lambda, \emptyset)$
  \State ByteList $S_0 \gets \mathbf{ParSerialize}(B)$
  \State \textbf{Write}($S_0$)
  \State OctreeNode $T_p \gets T_0$
  \ForEach {(PointSet $P_i : P$)} \Comment{as new $P_i$ stream in}
    \State OctreeNode $T_i \gets \mathbf{ParCreateOctree}(P_i)$
    \State \textbf{Write}($\vec{B}_i$)
    \State $B \gets $ThreadBlock$((T_p, T_i), \lambda, \emptyset)$
    \State $\delta_{p,i} \gets \mathbf{ParSymDiff}(B)$
    \State \textbf{Write}($\mathbf{RLE}(\delta_{p,i})$)
    \State $T_p \gets T_i$
  \EndFor
\EndProcedure
\Statex
\end{algorithmic}
\vspace{-0.4cm}
\end{algorithm}

%------------------------------------------------

\subsection{Parallel Decompression} \label{sec:Parallel Decompression}

Unfortunately, the other subroutines called upon during stream decompression cannot be efficiently
parallelized. Namely, \textbf{Algorithm \ref{alg:Deserialize}}, the subroutine to decode a serialized
octree, and \textbf{Algorithm \ref{alg:Reconstruct}}, the subroutine to reconstruct an octree given the
previous octree and their symmetric difference. The inability to extract parallelism from these subroutines
can be attributed to the nature of their most essential operations: performing computations while
traversing a linked list. This operation's inefficacy to be parallelized is due to its interdependence
of data from past, present, and future tasks' operations. Thus, those subroutines must execute sequentially.

However, when decompressing and an encoded point cloud stream, it is much more reasonable to make the
assumption that the entirety of the stream is initially available. This allows for more sophisticated
algorithms to be used during decompression. The algorithm we will propose to optimize point cloud
stream decompression makes use of a classic algorithm in parallel programming: Prefix sum \cite{prefix}

Before diving in to the complete algorithm, we must first define a new operator $+_M$ which will be used
to merge symmetric differences between octrees. Although in set theory, the symmetric difference operation
is a binary operator $\symdiff: \mathbb{S} \times \mathbb{S} \rightarrow \mathbb{S}$ that maps the relation
of two sets to another set, but the way we defined the symmetric difference operator with respect to octrees
does not map the relation two octrees to another octree, but rather an abstract $\delta$. This is because
in the calculation of symmetric difference, \textbf{Algorithm \ref{alg:CalcSymDiff}}, the mapping between
octrees is not surjective because unpopulated octree nodes are ignored to same space in the encoding.
Consequently, $\delta$ does not carry meaning on its own and must be used to build off of an existing
octree to yield another octree. Let $\Delta$ be the set of all possible symmetric differences $\delta$.
Therefore, let $+_M$ be a binary operator such that $+_M: \Delta \times \Delta \rightarrow \Delta$, and
$\delta_{i,j} +_M \delta_{j,k} = \delta_{i,k}$.

We now present an algorithm to compute $\delta_{i,j} +_M \delta_{j,k} = \delta_{i,k}$. Informally, the
algorithm consists of walking four byte lists; two copies of $\delta_{i,j}$ and two copies of
$\delta_{j,k}$. For each $\delta$, there is a parent list $p$ responsible for enqueuing the appropriate
byte, and a child list $c$ responsible for keeping a pointer at the next byte in the byte list to be
merged. When a bit is set in either parent byte, if they have a difference byte in there child list
corresponding to that set bit in the parent byte, a new byte is created which is the $\mathtt{XOR}$ of
any existing, corresponding child byte, which is then appended to the resulting byte list.

\alglanguage{pseudocode}
\begin{algorithm}[h]
\small
\caption{Merge two symmetric difference results}
\label{alg:Merge}
\begin{algorithmic}[1]
\Procedure {$\mathbf{Merge}$} {ByteList $\delta_{i,j}$, ByteList $\delta_{j,k}$}
  \State ByteList $\delta_{i,k}$
  \State ByteList $p_0, c_0 \gets \delta_{i,j}$
  \State ByteList $p_1, c_1 \gets \delta_{j,k}$
  \ForEach {(byte $b_0, b_1 : p_0, p_1$)}
    \ForEach {(set bit $x : b_0 \ \mathtt{OR} \ b_1$)}
      \State byte $r \gets 0$
      \If {($b_0[x]$ is set)}
        \State $r \gets r \oplus c_0$
        \State $c_0 \gets c_0.\mathtt{next}$
      \EndIf
      \If {($b_1[x]$ is set)}
        \State $r \gets r \oplus c_1$
        \State $c_1 \gets c_1.\mathtt{next}$
      \EndIf
      \State $\delta_{i,k}$.append($r$)
    \EndFor
  \EndFor
  \State \Return $\delta_{i,k}$
\EndProcedure
\Statex
\end{algorithmic}
\vspace{-0.4cm}
\end{algorithm}

% I think I need to prove this to be correct somehow, but I'm honestly lost on how to go about it...
We now argue why \textbf{Algorithm \ref{alg:Merge}} correctly merges the symmetric differences
$\delta_{i,j}$ and $\delta_{j,k}$ together to form $\delta_{i,k}$. Both byte lists are walked sequentially
in the order they were formed, meaning in the order an octree is traversed. Since two separate pointers
are used to walk each linked list, when a set bit is encountered, a new byte is inserted into the resulting
byte list in its proper order, because the octree is being traversed in the correct order. This new byte
either consists of a difference byte of a single difference list, if that bit was only set in one parent
list or it is the $\mathtt{XOR}$ of the two child difference bytes because if the tree is being traversed
in order, and a bit is set in the same place, then the child difference bytes must both correspond to the
octree node. Furthermore, if $T_i$ and $\delta_{i,k}$ were used to reconstruct $T_k$, then the
reconstruction would match the original tree, because the bits set in $\delta_{i,j}$ first toggle octree
nodes in $T_i$ yielding $T_j$, then $\delta_{j,k}$ toggles bits $T_j$ finally yielding the correct $T_k$.
Thus, $\delta_{i,j} +_M \delta_{j,k} = \delta_{i,k}$.

As explained in the beginning of this section, algorithms that mainly consist of walking linked lists like
\textbf{Algorithm \ref{alg:Merge}} are woefully condemned to sequential execution. However, this is
not a concern because a large amount of parallelism will be extracted via the next algorithm we present.

The final step before $+_M$ can be used in prefix sum is that the binary operator used in place of
addition must be associative. Hence, we must prove the associativity of $+_M$. An operator $\circ$ is
associative if $\forall a, b, c \in S$, we have that $a \circ (b \circ c) = (a \circ b) \circ c$. Let
$\circ$ be $+_M$, $S$ be $\Delta$, $a,b,c$ be $\delta_{i,j}, \delta_{j,k}, \delta_{k,l}$, respectively.
Based on the correctness of \textbf{Algorithm \ref{alg:Merge}}, we have that:
$$\delta_{i,j} +_M (\delta_{j,k} +_M \delta_{k,l}) = (\delta_{i,j} +_M \delta_{j,k}) +_M \delta_{k,l}$$
$$\delta_{i,j} +_M \delta_{j,l} = \delta_{i,k} +_M \delta_{k,l}$$
$$\delta_{i,l} = \delta_{i,l}$$
Therefore, $+_M$ is associative.

We now present the Prefix Merge algorithm, which is prefix sum with its operator being $+_M$ over the set
$\Delta$. Our end goal with Prefix Merge is to reclaim the most accurate representations of the original
point clouds given an encoded $T_0$ and $\forall i, 1 \leq i \leq n$, we have $\delta_{i-1, i}$.

Let $D[i], 1 \leq i \leq n$ be an array of symmetric differences $\delta_{i-1, i}$. By using the prefix
sum algorithm with $+_M$, we can transform $D$ such that $\forall i, D[i] = \delta_{0, i}$.

\alglanguage{pseudocode}
\begin{algorithm}[h]
\small
\caption{Compute the parallel prefix sum over $D$ using the binary operator $+_M$}
\label{alg:PrefixMerge}
\begin{algorithmic}[1]
\Procedure {$\mathbf{PrefixMerge}$} {ByteList$[\,]$ $D$}
  \State Integer $n \gets D.\mathtt{length}$
  \For {stride $\gets 1$ to $n$ by stride}
    \For {$i \gets$ stride to $n$ by stride$\times 2$}
      \For {$j \gets 1$ to stride}
        \State $\mathbf{spawn} \ D[i + j] \gets \mathbf{Merge}(D[i], D[i + j])$
      \EndFor
    \EndFor
    \State $\mathbf{sync}$
  \EndFor
  \State \Return $D$
\EndProcedure
\Statex
\end{algorithmic}
\vspace{-0.4cm}
\end{algorithm}

Using $D$, every $T_i$, and subsequently $P_i$, can be reconstructed in parallel by simultaneously invoking
$n$ instances of \textbf{Algorithm \ref{alg:Reconstruct}}, octree reconstruction, followed by assigning a
point to the centroid of each populated octree node in $T_i$, and writing the new point set as a .pcd file.
Therefore, our complete stream decompression algorithm is the following:

\alglanguage{pseudocode}
\begin{algorithm}[h]
\small
\caption{Decompress a compressed octree stream in parallel using Prefix Merge}
\label{alg:ParDecomp}
\begin{algorithmic}[1]
\Procedure {$\mathbf{ParDecomp}$} {OctreeNode $T_0$, ByteList$[\,]$ $D$}
  \State $D \gets \mathbf{PrefixMerge}(D)$
  \For {$i \gets 1$ to $n$}
    \State $\mathbf{spawn \ Write(ToPcd(Reconstruct}(T_0, D[i])))$
  \EndFor
  \State $\mathbf{sync}$
\EndProcedure
\Statex
\end{algorithmic}
\vspace{-0.4cm}
\end{algorithm}

%----------------------------------------------------------------------------------------

\section{Experimental Evaluation} \label{sec:Experimental Evaluation}

The current status of the implementation of these algorithms is as follows: The serial
elision is fully functional and tested, which implies the essential correctness of the
algorithms, and also establishes a benchmark that can be used to show the efficiency of 
the parallel versions. This will be shown further after the cilk \cite{cilk} and CUDA
implementations are completed.

%----------------------------------------------------------------------------------------

\section{Conclusion} \label{sec:Conclusion}

The efficiency of our parallel algorithms are theoretically more efficient than their
serial counterparts. More details will follow after the experimental evaluation is completed.

%----------------------------------------------------------------------------------------
%	REFERENCE LIST
%----------------------------------------------------------------------------------------

\begin{thebibliography}{15}

\bibitem{prefix}
Blelloch, Guy E. “Programming Parallel Algorithms.”
\textit{Communications of the ACM}, vol. 39, no. 3, 1996, pp. 85–97.

\bibitem{cilk}
Frigo, Matteo, et al. “The Implementation of the Cilk-5 Multithreaded Language.”
\textit{ACM SIGPLAN Notices}, vol. 33, no. 5, 1998, pp. 212–223.

\bibitem{trees} 
Jo, Youngjoon, et al. “Efficient Execution of Recursive Programs on Commodity Vector Hardware.”
\textit{ACM SIGPLAN Notices}, vol. 50, no. 6, 2015, pp. 509–520.

\bibitem{pcsc}
Kammerl, Julius, et al. “Real-Time Compression of Point Cloud Streams.”
\textit{2012 IEEE International Conference on Robotics and Automation}, 2012
 
\end{thebibliography}

%----------------------------------------------------------------------------------------

\end{document}
